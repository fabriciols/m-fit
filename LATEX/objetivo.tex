\section{Objetivo \label{objetivo}}

\hspace*{1.25cm}Desenvolver um sistema autom�tico de detec��o de
transi��o de tomadas e edi��o de v�deo.

\section{Contribui��es do Trabalho \label{contribuicao}}

\hspace*{1.25cm}O presente trabalho de conclus�o de curso apresenta
as seguintes contribui��es:

\begin{itemize}

\item{Estudo e implementa��o de m�todos para detec��o de
transi��es de v�deos.}

\item{Um modelo UML que serve como base para um sistema de
editora��o eletr�nica de v�deos baseado nas transi��es encontradas.}

\end{itemize}

\section{Estrutura da Monografia \label{estrutura}}

\hspace*{1.25cm}O restante do trabalho est� dividido da seguinte
maneira:

\hspace*{0.65cm}O Cap�tulo 2 apresenta alguns trabalhos publicados
na literatura cient�fica relacionados ao assunto desta monografia;

\hspace*{0.65cm}No Cap�tulo 3 s�o apresentados os conceitos b�sicos
necess�rios para entend�-la;

\hspace*{0.65cm}No Cap�tulo 4 � apresentada a metodologia proposta;

\hspace*{0.65cm}Os Cap�tulos 5 e 6 apresentam a especifica��o de
requisitos e a \\modelagem do sistema M-FIT.

\hspace*{0.65cm}No Cap�tulo 7 � mostrada uma pr�via de como ser� a
interface do sistema.
