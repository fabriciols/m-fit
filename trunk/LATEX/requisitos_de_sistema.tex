\section{Requisitos de Sistema \label{RQS}}

\subsection{Localiza��o de arquivo \label{RQS01}}

O sistema dever� ser capaz de abrir arquivos que estejam em qualquer
tipo de disco de armazenamento, pen drives, HD, CD�s, entre outros.

\subsection{Montagem do Ritmo Visual \label{RQS02}}

Aplica��o de c�lculos para extra��o da diagonal principal de todos
os frames do v�deo e transforma��o em uma coluna do ritmo visual.

\subsection{Montagem de mapa de bordas \label{RQS03}}

Aplicar um filtro passa-baixa, eliminando os ru�dos do Ritmo Visual.
Com isso o sistema obter� o mapa de bordas do Ritmo Visual.

\subsection{Tipos de v�deo \label{RQS04}}

O sistema dever� ser compat�vel no m�nimo com v�deos do tipo AVI e
MPEG.

\subsection{Sistema Operacional \label{RQS05}}

O sistema dever� ser desenvolvido para rodar na plataforma Windows,
por�m deve-se visar a portalidade do sistema.

\subsection{Digitaliza��o do v�deo \label{RQS06}}

O v�deo fornecido ao sistema dever� ter sido anteriormente
digitalizado pelo usu�rio, n�o cabendo ao sistema esta fun��o.

\subsection{OpenCV \label{RQS07}}

Necess�ria a inclus�o da biblioteca OpenCV para que o sistema possa
utilizar fun��es de PDI.

\subsection{QT \label{RQS08}}

A interface do sistema dever� ser desenvolvida atrav�s da ferramenta
QT.
