\subsection{Cen�rio Edi��o \label{use_case_edicao}}

\begin{figure}[h|top]
 \centering
 \includegraphics[width=1.0\linewidth]{imagens/use_case_edicao.png}
 \caption{Caso de Uso para cen�rio de edi��o de v�deo.}
 \label{img_use_case_edicao}
\end{figure}

A Figura \ref{img_use_case_edicao}

\subsubsection{Caso de uso: Aplica efeitos \label{use_case_aplica_efeito}}

\textbf{Cen�rio Principal:} O usu�rio aplica um ou mais efeitos de
edi��o no v�deo carregado.

\subsubsection{Caso de uso: Define intervalo de aplica��o \label{use_case_define_aplic}}

\textbf{Cen�rio Principal:} O usu�rio deve definir qual ser� o
trecho do v�deo em que dever� ser aplicado o efeito selecionado. Por
default, o sistema aplica o efeito na tomada que estiver
selecionada, ou no v�deo todo, caso n�o haja nenhuma tomada
selecionada.

\subsubsection{Caso de uso: Equaliza ilumina��o \label{use_case_iluminacao}}

\textbf{Cen�rio Principal:} O usu�rio poder� realizar a equaliza��o
da ilumina��o em uma tomada do v�deo. A qual poder� ser autom�tica,
ou frame a frame.

\subsubsection{Caso de uso: Ajuste de Contraste \label{use_case_contraste}}

\textbf{Cen�rio Principal:} O usu�rio poder� realizar ajustes de
contraste em uma tomada do v�deo.

\subsubsection{Caso de uso: Altera��o do Esquema de Cores \label{use_case_cores}}

\textbf{Cen�rio Principal:} O usu�rio aplica efeitos para altera��o
do esquema de cores do v�deo carregado ou de uma tomada espec�fica
do v�deo.
