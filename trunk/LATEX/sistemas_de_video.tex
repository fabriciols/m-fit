\section{Sistemas de V�deo \label{sistemas_de_video}}

Os sistemas de v�deo consistem na organiza��o em que uma determinada
informa��o de v�deo � armazenada em forma digital, tais como AVI,
MPEG, entre outros. O MPEG (Movie Pictures Experts Group) � um
padr�o internacional definido pela ISO que tem como caracter�stica a
compress�o de um v�deo. Onde ao comprimir uma determinada amostragem
de um v�deo, o mesmo � passado pelo canal compressor, e ao exibir
esta mesma amostragem o v�deo passa pelo canal expansor.

\begin{figure}[h|top]
 \centering
 \includegraphics[width=0.7\linewidth]{imagens/mpeg.png}
 \caption{Sistema de Codifica��o e Decodifica��o de um v�deo no formato MPEG.}
 \label{img_mpeg}
\end{figure}


Audio Video Interleave, o AVI � um padr�o criado pela Microsoft em
1992, formato derivado do padr�o RIFF, o qual divide os dados em
blocos. Ao contr�rio dos outros padr�es o AVI n�o possui compress�o,
resultando em arquivos com tamanhos grandes, por�m, sem perda de
qualidade. Na Figura 2 observa-se que o arquivo AVI � formado
primeiramente pelo cabe�alho RIFF seguido pelo tamanho do bloco que
ser� armazenado fisicamente no disco, isto �, o tamanho total de um
arquivo AVI. Em seguida o bloco AVI � definido com sua respectiva
lista de sub-blocos, que s�o identificados a partir de um ID e seu
tamanho correspondente.

\begin{figure}[h|top]
 \centering
 \includegraphics[width=0.3\linewidth]{imagens/avi.png}
 \caption{Sistema de armazenamento AVI.}
 \label{img_avi}
\end{figure}
