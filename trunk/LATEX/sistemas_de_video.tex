\section{Sistemas de V�deo \label{sistemas_de_video}}

\hspace*{1.25cm}Os sistemas de v�deo consistem na organiza��o em que
uma determinada informa��o de v�deo � armazenada em forma digital,
tais como AVI, MPEG, entre outros. O MPEG (Movie Pictures Experts
Group) � um padr�o internacional definido pela ISO que tem como
caracter�stica a compress�o de um v�deo, onde, ao comprimir uma
determinada amostragem de um v�deo, o mesmo � passado pelo canal
compressor e, ao exibir esta mesma amostragem, o v�deo passa pelo
canal expansor (Figura \ref{img_mpeg}).

\begin{figure}[h|top]
 \centering
 \includegraphics[width=1.0\linewidth]{imagens/mpeg.png}
 \caption{Sistema de Codifica��o e Decodifica��o de um v�deo no formato MPEG.}
 \label{img_mpeg}
\end{figure}


\hspace*{0.65cm}Por outro lado, o AVI (Audio Video Interleave) � um
padr�o criado pela Microsoft em 1992, formato derivado do padr�o
RIFF (Resource Interchange File Format), o qual divide os dados em
blocos. Ao contr�rio dos outros padr�es, o AVI n�o possui
compress�o, resultando em arquivos com tamanhos grandes, por�m, sem
perda de qualidade. Na Figura \ref{img_avi} observa-se que, da
direita para a esquerda, a estrutura de dados de um arquivo AVI
primeiramente � formado pelo cabe�alho RIFF, onde seus respectivos
componentes s�o: tamanho do arquivo total, bloco de dados ou lista
de bloco de dados. Quando o bloco de dados � definido em sub-blocos,
cada parte � identificado pelo ID e em seguida o tamanho do
sub-bloco correspondente.

\hspace*{0.65cm}Neste trabalho ser�o usados somente videos AVI, por
serem os mais populares e f�ceis de encontrar exemplos para teste do
sistema M-FIT.

\begin{figure}[h|top]
 \centering
 \includegraphics[width=0.8\linewidth]{imagens/avi.png}
 \caption{Sistema de armazenamento AVI.}
 \label{img_avi}
\end{figure}
