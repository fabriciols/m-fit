\section{CONCLUS�O \label{conclusao}}

\hspace*{1.25cm}Neste trabalho foram apresentadas t�cnicas de
detec��o de transi��o baseadas em diversas metodologias e para cada
tipo de transi��o foi adotada uma metodologia espec�fica. Foi
observado durante os testes de detec��o de cortes que dependendo das
imagens presentes no v�deo, � necess�rio adequar o filtro de Canny
para que seja feita uma detec��o adequada, visto que os resultados
obtidos a partir da aplica��o deste filtro nas imagens variam
conforme o comportamento da lumin�ncia dos frames em an�lise. Os
valores aqui utilizados como limiar ficam vulner�veis � transi��es
entre duas tomadas que sejam mais escuras, por�m se estes limiares
forem configurados para detectar este tipo de transi��o, pode
ocorrer o aumento de falsas detec��es devido ao aumento de ru�do no
mapa de bordas. Desta forma, optou-se por utilizar limiares mais
altos para que as transi��es entre tomadas mais claras fossem
detectadas, visto que estas ocorrem em maioria.
