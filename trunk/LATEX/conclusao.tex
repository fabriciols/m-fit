\chapter{CONCLUS�O \label{conclusao}}

\hspace*{1.25cm}Neste trabalho foram apresentadas t�cnicas de
detec��o de transi��o baseadas em diversas metodologias e para cada
tipo de transi��o foi adotada uma metodologia espec�fica. Foi
observado durante os testes de detec��o de cortes que a metodologia
utilizada gerou resultados bastante aceit�veis em n�veis de falsos
positivos e falsos negativos, chegando a atingir a perfei��o em
alguns casos espec�ficos. Por�m, a metodologia mostra-se vulner�vel
em rela��o a cortes que ocorram entre duas tomadas cuja luminosidade
seja muito baixa. Nos testes realizados para detec��o de transi��es
do tipo fade, foi poss�vel observar que a metodologia utilizada
atinge resultados satisfat�rios, obtendo em sua maioria, baixos
�ndices de falsos positivos. Por outro lado, quando as transi��es do
tipo fade se apresentam de forma extremamente curta, estas s�o
facilmente confundidas com cortes, aumentando o n�mero de falsos
positivos.
