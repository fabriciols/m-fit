\chapter*{\centering RESUMO \label{resumo}}

\hspace*{1.25cm}Na �rea de Editora��o de V�deos, uma das maiores
demandas � a detec��o de transi��o de tomadas, que � basicamente o
corte de uma tomada e o inicio de outra. Essa tarefa tem v�rias
utilidades em diversas aplica��es, como indexa��o, recupera��o,
an�lise e edi��o de v�deos. Estas transi��es podem ocorrer de
diversas formas, tais como: cortes, fades ou dissolve. Este trabalho
apresenta o M-FIT Motion Finder Integration, um framework que
possibilitar� o reconhecimento autom�tico das transi��es de tomadas
de cenas a partir de uma determinada amostragem de imagens (frames
de um filme) atrav�s do uso de t�cnicas de Ritmo Visual e Morfologia
Matem�tica, permitindo que o ponto exato da transi��o seja obtido de
uma maneira mais pr�tica e r�pida. O M-FIT � uma ferramenta
desenvolvida em linguagem C, com utiliza��o de OpenCV (biblioteca
para computa��o visual). Sua interface � baseada em QT, tornando o
M-FIT um sistema com uma interface simples e de f�cil entendimento,
a qual ter� diversos atalhos para outras funcionalidades do sistema.
