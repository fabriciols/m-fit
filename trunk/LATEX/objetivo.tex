\section{Objetivo \label{objetivo}}

\hspace*{1.25cm}Desenvolver um sistema autom�tico de detec��o de
transi��o de tomadas e edi��o de v�deo.

\section{Contribui��o da Monografia \label{contribuicao}}

\hspace*{1.25cm}O presente trabalho de conclus�o de curso apresenta
as seguintes contribui��es:

\begin{itemize}

\item{Estudo e implementa��o de m�todos para detec��o de
transi��es de v�deos.}

\item{Um modelo UML que serve como base para um sistema de
editora��o eletr�nica de v�deos baseado nas transi��es encontradas.}

\item{Apresenta��o te�rica de t�cnicas modernas de detec��o de transi��o.}

\section{Estrutura da Monografia \label{estrutura}}

\hspace*{1.25cm}O restante do trabalho est� dividido da seguinte
maneira:

\hspace*{1.25cm}O cap�tulo 2 apresenta alguns dos principais
trabalhos publicados na literatura cient�fica relacionados ao
assunto desta monografia;

\hspace*{1.25cm}No cap�tulo 3 s�o apresentados os conceitos b�sicos
necess�rios para entend�-la;

\hspace*{1.25cm}No cap�tulo 4 � apresentada a metodologia proposta;

\hspace*{1.25cm}Finalmente, os cap�tulos 5 e 6 apresentam a
especifica��o de resquisitos e a modelagem do sistema M-FIT.
